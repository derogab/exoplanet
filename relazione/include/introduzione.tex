\chapter{Introduzione}
Durante la realizzazione di questo progetto ci siamo proposti di studiare 
alcuni dei modelli per il riconoscimento di esopianeti. 

Il dataset da noi scelto per le analisi è messo a disposizione dalla 
NASA\footnote{\href{https://www.kaggle.com/nasa/kepler-exoplanet-search-results}
{https://www.kaggle.com/nasa/kepler-exoplanet-search-results}} e si tratta di un
aggregatore di dati che sono stati raccolti dal telescopio \textbf{Kepler Space Telescope} 
per lo studio e la ricerca di esopianeti. \\
Nel dettaglio sono fornite una serie di informazioni relative a varie osservazioni
selezionate come potenziali esopianeti che successivamente vengono quindi 
classificate come tali o meno; nel dettaglio è a loro 
applicata l'etichetta di \textit{confermato}, \textit{falso positivo} o, se ancora 
in fase di studio, \textit{candidato} o \textit{non classificato}.

Si è quindi iniziato con uno studio esplorativo del dataset per poi passare all'utilizzo
e allo studio di alcuni modelli di machine learning.
