\chapter{Modelli di Machine Learning}
Per lo sviluppo dei modelli di Machine Learning è stato utilizzato
\textit{Caret}, un pacchetto contenente un insieme di funzioni in grado di 
creare modelli predittivi ed effettuare il train su di essi.

Sono stati scelti:
\begin{itemize}
    \item Support Vector Machine
    \item NaiveBayes
    \item Reti Neurali
\end{itemize}

Le operazioni di train di tutti i modelli utilizzati sono state effettuate 
per mezzo della crossfold validation. Anche per questa esecuzione si sono utilizzati 
gli strumenti messi a disposizione dal pacchetto \textit{Caret}.

Si è scelto inoltre di scalare i dati in modo che le predizioni eseguite non siano 
influenzate dai diversi tipi di misure presenti nel dataset.

La scelta di questi modelli è stata presa con la volontà di testare i vari
modelli di apprendimento supervisionato.
Da tale considerazione è stato però necessario trascurare gli alberi di 
decisione in quanto la natura dei valori ne rendeva praticamente impossibile 
l'uso.

\section{Support Vector Machine}
Per questo modello si è scelto di sfruttare due diverse tipologie di Kernel
per provare a studiare eventuali miglioramenti prestazionali. 
\subsection{Kernel Radiale}

\begin{figure}[H]
    \centering
    \includegraphics[width = \textwidth]{../outputs/confusion_matrix_svmRadial.comparison.png}
    \caption{Fourfoldplot per la matrice di confusione ottenuta con svm e kernel radiale}
\end{figure}
\begin{figure}[H]
    \centering
    \includegraphics[width = \textwidth]{../outputs/roc_svm_radial.comparison.png}
    \caption{Grafico con la ROC ottenuta con svm e kernel radiale}
\end{figure}
Si ottengono con questo modello:
\begin{itemize}
    \item accuracy
    \item precision
    \item recall
    \item f-measure
\end{itemize}
% ROC singola

\subsection{Kernel Polinomiale}
\begin{figure}[H]
    \centering
    \includegraphics[width = \textwidth]{../outputs/confusion_matrix_svmPolynomial.comparison.png}
    \caption{Fourfoldplot per la matrice di confusione ottenuta con svm e kernel polinomiale}
\end{figure}
\begin{figure}[H]
    \centering
    \includegraphics[width = \textwidth]{../outputs/roc_svm_polynomial.comparison.png}
    \caption{Grafico con la ROC ottenuta con svm e kernel polinomiale}
\end{figure}
Si ottengono con questo modello:
\begin{itemize}
    \item accuracy
    \item precision
    \item recall
    \item f-measure
\end{itemize}
\section{NaiveBayes}
\begin{figure}[H]
    \centering
    \includegraphics[width = \textwidth]{../outputs/confusion_matrix_bayes.comparison.png}
    \caption{Fourfoldplot per la matrice di confusione ottenuta con naive bayes}
\end{figure}
\begin{figure}[H]
    \centering
    \includegraphics[width = \textwidth]{../outputs/roc_bayes.comparison.png}
    \caption{Grafico con la ROC ottenuta con naive bayes}
\end{figure}
Si ottengono con questo modello:
\begin{itemize}
    \item accuracy
    \item precision
    \item recall
    \item f-measure
\end{itemize}


\section{Reti Neurali}
\begin{figure}[H]
    \centering
    \includegraphics[width = \textwidth]{../outputs/confusion_matrix_network.comparison.png}
    \caption{Fourfoldplot per la matrice di confusione ottenuta con la rete neaurale}
\end{figure}
\begin{figure}[H]
    \centering
    \includegraphics[width = \textwidth]{../outputs/roc_network.comparison.png}
    \caption{Grafico con la ROC ottenuta con la rete neaurale}
\end{figure}
Si ottengono con questo modello:
\begin{itemize}
    \item accuracy
    \item precision
    \item recall
    \item f-measure
\end{itemize}
