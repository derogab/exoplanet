\chapter{Analisi del dataset}
Come anticipato, il dataset fornisce informazioni in merito alle varie 
osservazioni ottenute dal \textbf{Kepler Space Telescope} dei cosiddetti
\textbf{Kepler Object of Interest (\textit{KOI})}.\\

La NASA fornisce una descrizione di ogni colonna presente nel dataset e 
pubblicamente accessibile a 
\href{https://exoplanetarchive.ipac.caltech.edu/docs/API_kepcandidate_columns.html}{questo link}.

\section{Analisi dei dati}

\subsection{Target}

La colonna target è rappresantata da \textbf{koi\_disposition}, la quale 
fornisce l'indicazione in merito alla classificazione dell'oggetto studiato.
Come è già stato anticipato, questa colonna può assumere i seguenti 4 valori: 
\begin{itemize}
    \item \textbf{CONFIRMED}, per indicare che quell'osservazione ha portato al 
    riconoscimento effettivo dell'esopianeta.
    \item \textbf{FALSE POSITIVE}, per indicare che quell'osservazione è stata 
    riconosciuta come un falso positivo, non indicando quindi un esopianeta.
    \item \textbf{CANDIDATE}, per indicare che la comunità scientifica non si 
    è ancora espressa in merito alla natura dell'osservazione in quanto mancano
    ancora diversi test sulle osservazioni, nonostante siano stati fatti già
    i test che escludano l'osservazione dall'essere catalogata come 
    \textit{FALSE POSITIVE}.
    \item \textbf{NOT DISPOSITIONED}, per indicare che non sono stati ancora 
    eseguiti nemmeno i test he escludano l'osservazione dall'essere catalogata 
    come \textit{FALSE POSITIVE}.
\end{itemize}

\subsection{Altre informazioni sull'archivio degli esopianeti}
Proseguendo l'analisi del dataset individuiamo altre colonne 
che identificano l'osservazione attraverso un numero incrementale ed una 
identificativo alfanumerico, assegnati a priori alle osservazioni. 
\begin{itemize}
    \item kepid
    \item kepoi\_name
\end{itemize}

\subsection{Informazioni sui dati delle osservazioni}
Si ha anche, per le osservazioni che hanno portato all'identificazione effetiva di un 
esopianeta, l'indicazione del nome assegnato al corpo celeste. Queste 
informazioni sono in generale riconducibili quindi a due categorie:
\begin{itemize}
    \item \textbf{Identification Columns}
    \item \textbf{Exoplanet Archive Information}
\end{itemize}
Si ha poi la categoria \textbf{Project Disposition Columns} con varie 
informazioni relative alle osservazioni. Tra esse si distinguono i valori di 
\textbf{koi\_pdisposition} che, a differenza di analizzare lo stato attuale 
dello studio delle osservazioni (come in \textbf{koi\_disposition}), 
indicano le classi che potrebbero essere 
assegnate in modo probabilistico a partire dai dati.\\
Si ha quindi, per esempio, un'indicazione dello score indicante la 
confidenza della \textbf{koi\_disposition}.\\
Si hanno anche quattro flag booleani che vale la pena analizzare:
\begin{enumerate}
    \item \textbf{Not Transit-Like Flag (\textit{koi\_fpflag\_nt})} indicante
    che il KOI ha una curva di luce non è coerente con quella di un pianeta in 
    transito (anche se questo potrebbe essere ricondotto ad errori nella 
    strumentazione).
    \item \textbf{Stellar Eclipse Flag (\textit{koi\_fpflag\_ss})} indicante
    che il KOI ha l'evento simile al transito potrebbe essere causato da un 
    sistema stellare binario (anche un eventuale \textit{gioviano caldo} 
    potrebbe avere questo flag settato)
    \item \textbf{Centroid Offset Flag (\textit{koi\_fpflag\_co})} indicante
    che il sorgente del segnale proviene da una stella vicina
    \item \textbf{Ephemeris Match Indicates Contamination Flag
    (\textit{koi\_fpflag\_ec})} indicante che il KOI condivide lo stesso periodo
    e l'epoca di un altro oggetto e che viene ritenuto essere il risultato di
    una "contaminazione" durante l'analisi
\end{enumerate}
Le successive colonne indicano invece vari dati non booleani relative alle varie 
misurazioni effettuate (misurazioni eventualmente arricchite con errori). Nel 
dettaglio si riconsocono varie categorie:
\begin{itemize}
    \item \textbf{Transit Properties}, con varie informazioni in merito al 
    transito del KOI
    \item \textbf{Threshold-Crossing Event Information}, ...
    \item \textbf{Stellar Parameters}, con informaizoni relative alla stella
    associata al KOI
    \item \textbf{KIC Parameters}, ...
    \item \textbf{Pixel-Based KOI Vetting Statistics}, ...
\end{itemize}














In altri termini le prime due saranno le etichette che useremo per le 
previsioni, che quindi saranno di tipo binario. Si è deciso di escludere la 
classe relativa a \textit{CANDIDATE} in quanto non si potrebbe avere un 
riscontro effettivo con la realtà approvata dalla comunità scientifica di 
un'ipotetica predizione tramite un modello di machine learning. Si è quindi
deciso di rimuovere dal dataset tutte le righe aventi l'etichetta 
\textit{CANDIDATE}.\\
Una seconda osservazione è in merito al \textbf{downsample}. Il dataset 
presentava 


Una prima considerazione deve essere fatta in merito a quelle informazioni che
implicitamente sono ricollegabili all'etichetta assegnata.





\section{PCA}