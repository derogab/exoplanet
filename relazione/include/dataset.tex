\chapter{Analisi del dataset}
Come anticipato, il dataset fornisce informazioni in merito alle varie 
osservazioni ottenute dal \textbf{Kepler Space Telescope} dei cosiddetti
\textbf{Kepler Object of Interest (\textit{KOI})}.\\

La NASA fornisce una descrizione di ogni colonna presente nel dataset e 
pubblicamente accessibile a 
\href{https://exoplanetarchive.ipac.caltech.edu/docs/API_kepcandidate_columns.html}{questo link}.

\section{Analisi dei dati}

La colonna \textit{target} è rappresantata da \textbf{koi\_disposition}, la quale 
fornisce l'indicazione in merito alla classificazione dell'oggetto studiato.
Come è già stato anticipato, questa colonna può assumere i seguenti 4 valori: 
\begin{itemize}
    \item \textbf{CONFIRMED}, per indicare che quell'osservazione ha portato al 
    riconoscimento effettivo dell'esopianeta.
    \item \textbf{FALSE POSITIVE}, per indicare che quell'osservazione è stata 
    riconosciuta come un falso positivo, non indicando quindi un esopianeta.
    \item \textbf{CANDIDATE}, per indicare che la comunità scientifica non si 
    è ancora espressa in merito alla natura dell'osservazione in quanto mancano
    ancora diversi test sulle osservazioni, nonostante siano stati fatti già
    i test che escludano l'osservazione dall'essere catalogata come 
    \textit{FALSE POSITIVE}.
    \item \textbf{NOT DISPOSITIONED}, per indicare che non sono stati ancora 
    eseguiti nemmeno i test he escludano l'osservazione dall'essere catalogata 
    come \textit{FALSE POSITIVE}.
\end{itemize}

Proseguendo l'analisi del dataset individuiamo alcune colonne 
che identificano l'osservazione con dei valori assegnati a priori,
come un numero incrementale ed un identificativo alfanumerico.

Si ha anche, per le osservazioni che hanno portato all'identificazione effettiva di un 
esopianeta, l'indicazione del nome assegnato al corpo celeste. 

Le suddette informazioni sono generalmente riconducibili a due categorie di dati:
\begin{itemize}
    \item \textbf{Identification Columns}
    \item \textbf{Exoplanet Archive Information}
\end{itemize}

Un ulteriore gruppo di dati è rappresentato dalla categoria 
\textbf{Project Disposition Columns}, dove troviamo ulteriori informazioni 
riguardanti l'osservazione.

Nel dettaglio abbiamo il valore \textbf{koi\_pdisposition} che indica le classi 
che potrebbero essere assegnate in modo probabilistico a partire dai dati 
(a differenza di \textbf{koi\_disposition} che analizza lo stato attuale 
dello studio delle osservazioni).

Si ha inoltre un'indicazione (\textbf{koi\_score}) indicante lo score della 
confidenza del valore presente in \textbf{koi\_disposition}.

Nella medesima categoria si trovano anche quattro flag booleani che vale la 
pena analizzare in modo approfondito:
\begin{enumerate}
    \item \textbf{Not Transit-Like Flag (\textit{koi\_fpflag\_nt})} indicante
    che il KOI ha una curva di luce non è coerente con quella di un pianeta in 
    transito (anche se questo potrebbe essere ricondotto ad errori nella 
    strumentazione).
    \item \textbf{Stellar Eclipse Flag (\textit{koi\_fpflag\_ss})} indicante
    che il KOI ha l'evento simile al transito potrebbe essere causato da un 
    sistema stellare binario (anche un eventuale \textit{gioviano caldo} 
    potrebbe avere questo flag settato).
    \item \textbf{Centroid Offset Flag (\textit{koi\_fpflag\_co})} indicante
    che il sorgente del segnale proviene da una stella vicina.
    \item \textbf{Ephemeris Match Indicates Contamination Flag
    (\textit{koi\_fpflag\_ec})} indicante che il KOI condivide lo stesso periodo
    e l'epoca di un altro oggetto e che viene ritenuto essere il risultato di
    una "contaminazione" durante l'analisi.
\end{enumerate}

Tutti i dati delle colonne seguenti rappresentano valori non booleani relativi
alle effettive misurazioni (eventualmente arricchite con errori) di una 
certa osservazione. 

Nel dettaglio si riconsocono varie categorie tra cui informazioni in merito al 
transito,a i dati della stella etc$\ldots$

\section{Selezione iniziale dei dati}
Dopo una prima analisi del dataset si è scelto di modificarlo al fine di poter 
ottenere un dataset più pulito, con le sole indicazioni utili ai fini dello 
studio.\\
Innanzitutto si è deciso di concentrarsi unicamente sulle etichette 
\textbf{CONFIRMED} e \textbf{FALSE POSITIVE} in quanto le altre due categorie
non permetterebbero una corretta analisi delle predizioni dei modelli di machine
learning, non avendo un'opinione "sicura" della comunità scientifica in merito 
ai KOI. Il problema viene quindi ridotto ad un caso binario.\\
Proseguendo si è ovviamente scelto di rimuovere le colonne relative a nomi ed
identificatori (che in alcuni casi sono direttamente legati alla label 
\textbf{CONFIRMED}) dei KOI. Viene rimosso anche lo score di punteggio della 
confidenza. \\
Dal punto di vista dei quattro booleani trattati precedentemente viene scelto
di non tenere nessuno dei quattro attributi. Questa scelta è dovuta al fatto 
che, nonostante, come descritto, lascino spazio per eventuali riconoscimenti 
sia positivi che negativi, sono prettamente legati, a seconda del flag, ad una
certa label. \\
Il dataset forniva inoltre un attributo \textit{koi\_vet\_stat} (con associata 
anche la data) che certifica, con il valore \textit{Done}, che la comunità 
scientifica è giunta ad una conclusione ufficiale sul KOI. Viene scelto, per
poter avere un confronto più sicuro tra gli esiti dei modelli e i dati 
ufficiali, di lavorare quindi solo con tali valori (trascurando quindi i 
record con tale attributo a valore \textit{Active}). La colonna relativa alla 
data di validazione viene ovviamente rimossa.\\
Per lavorare con un dataset più pulito si è anche scelto di eliminare le colonne
relative ad attributi in cui l'intero dataset presentava, per ogni riga, il solo
valore \textit{NULL}.\\
Vengono rimosse anche le poche colonne con variabili categoriche.\\

Una seconda osservazione è in merito al \textbf{downsample}. Il dataset 
presentava uno sbilanciamento nei confronti dei KOI \textbf{FALSE POSITIVE}. Si 
è quindi scelto, non avendo motivo di dare più peso a questi KOI, di procedere 
con il downsample.


\begin{figure}[!htb]
    \begin{minipage}{0.48\textwidth}
      \centering
      \includegraphics[width=.7\linewidth]{../outputs/before_downsample_distr.pca.png}
      \caption{Distribuzione del target prima il downsampling}
      \label{Fig:Data1}
    \end{minipage}\hfill
    \begin{minipage}{0.48\textwidth}
      \centering
      \includegraphics[width=.7\linewidth]{../outputs/after_downsample_distr.pca.png}
      \caption{Distribuzione del target dopo il downsampling}
      \label{Fig:Data2}
    \end{minipage}
\end{figure}
% immagine matrice correlazione

\section{Principal Component Analysis}
Si è quindi proceduto con la PCA sul dataset intero. È stato scelto di tenere 
le feature per le quali, con la tecnica della PCA, si ottiene un autovalore
maggiore o uguale a 1 in quanto è una delle metriche maggiormente utilizzate.
\begin{figure}[H]
    \centering
    \includegraphics[width = \textwidth]{../outputs/hist_pca.png}
    \caption{Barplot dei contributi delle variabili rispetto alle dimensioni 
    della PCA. La linea rossa rappresenta il contributo medio atteso.}
\end{figure}
%\begin{figure}
%    \centering
%    \includegraphics[width = \textwidth]{../outputs/distr.pca.png}
%    \caption{Screeplot con gli autovalori ordinati dal maggiore al minore}
%\end{figure}
\begin{figure}[H]
    \centering
    \includegraphics[width = .7\textwidth]{../outputs/cumulative_variance.pca.png}
    \caption{Plot della varianza cumulata delle componenti della PCA} 
\end{figure}
Alla fine della PCA sono state quindi selezionate 26 feature più significative,
a partire dalle iniziali 100, che verranno utilizzate per i modelli di machine 
learning. Queste numero di variabili comportano una varianza cumulata di 
$\sim 85\%$.