\chapter{Analisi del dataset}
Come anticipato, il dataset fornisce informazioni in merito alle varie 
osservazioni ottenute dal \textbf{Kepler Space Telescope} dei cosiddetti
\href{https://en.wikipedia.org/wiki/Kepler_object_of_interest}
{\textbf{Kepler Object of Interest (\textit{KOI})}}, ovvero stelle per le quali
si presuppone l'esistenza di un esopianeta nel loro sistema stellare.\\

La NASA fornisce una descrizione di ogni colonna presente nel dataset e 
pubblicamente accessibile a nella documentazione ufficiale
\footnote{\href{https://exoplanetarchive.ipac.caltech.edu/docs/API_kepcandidate_columns.html}
{https://exoplanetarchive.ipac.caltech.edu/docs/API\_kepcandidate\_columns.html}}.

Inizialmente è stato utilizzato il dataset disponibile su kaggle. In seguito tuttavia ci 
siamo accorti che non tutte le colonne descritte dalla documentazione ufficiale della NASA erano presenti. 
È stato quindi scaricato ed utilizzato il dataset completo, presente nell'archivio ufficiale della NASA
\footnote{\href{https://exoplanetarchive.ipac.caltech.edu/cgi-bin/TblView/nph-tblView?app=ExoTbls&config=koi}
{https://exoplanetarchive.ipac.caltech.edu/cgi-bin/TblView/nph-tblView?app=ExoTbls\&config=koi}}.


\section{Analisi dei dati}

La colonna \textit{target} è rappresantata da \textbf{koi\_disposition}, 
capace di fornire l'indicazione in merito alla classificazione dell'oggetto studiato.
Come è già stato anticipato, questa colonna può assumere i seguenti 4 valori: 
\begin{itemize}
    \item \textbf{CONFIRMED}, per indicare che quell'osservazione ha portato al 
    riconoscimento effettivo dell'esopianeta.
    \item \textbf{FALSE POSITIVE}, per indicare che quell'osservazione è stata 
    riconosciuta come un falso positivo, non indicando quindi un esopianeta.
    \item \textbf{CANDIDATE}, per indicare che la comunità scientifica non si 
    è ancora espressa in merito alla natura dell'osservazione in quanto mancano
    ancora diversi test sulle osservazioni, nonostante siano stati fatti già
    i test che escludano l'osservazione dall'essere catalogata come 
    \textit{FALSE POSITIVE}.
    \item \textbf{NOT DISPOSITIONED}, per indicare che non sono stati ancora 
    eseguiti nemmeno i test he escludano l'osservazione dall'essere catalogata 
    come \textit{FALSE POSITIVE}.
\end{itemize}

Proseguendo l'analisi del dataset, individuiamo anche alcune colonne 
contenenti degli identificativi dati a priori alle osservazioni, nel dettaglio 
un numero incrementale ed un identificativo alfanumerico.

In modo similare, per le sole osservazioni che hanno portato all'identificazione 
effettiva di un esopianeta, troviamo l'indicazione del nome assegnato al corpo celeste. 

Le suddette informazioni sono generalmente riconducibili a due categorie di dati:
\begin{itemize}
    \item \textbf{Identification Columns}
    \item \textbf{Exoplanet Archive Information}
\end{itemize}

Un successivo gruppo di dati è invece rappresentato dalla categoria 
\textbf{Project Disposition Columns}, dove troviamo ulteriori informazioni 
riguardanti l'osservazione.

Nel dettaglio abbiamo il valore \textbf{koi\_pdisposition} indicante la 
classificazione stimata in modo probabilistico per gli oggetti per cui 
non si è ancora ottenuta una conferma scientifica.
In allegato si ha anche il valore di score (\textbf{koi\_score})
indicante l'effettivo valore di probabilità della classificazione presente in
\textit{koi\_pdisposition}.

Nella medesima categoria si possono trovare anche quattro flag booleani di cui 
si necessita una analisi più approdondita:
\begin{enumerate}
    \item \textbf{Not Transit-Like Flag (\textit{koi\_fpflag\_nt})} indicante
    che il KOI ha una curva di luce non è coerente con quella di un pianeta in 
    transito (anche se questo potrebbe essere ricondotto ad errori nella 
    strumentazione).
    \item \textbf{Stellar Eclipse Flag (\textit{koi\_fpflag\_ss})} indicante
    che il KOI ha l'evento simile al transito potrebbe essere causato da un 
    sistema stellare binario (anche un eventuale \textit{gioviano caldo} 
    potrebbe avere questo flag settato).
    \item \textbf{Centroid Offset Flag (\textit{koi\_fpflag\_co})} indicante
    che il sorgente del segnale proviene da una stella vicina.
    \item \textbf{Ephemeris Match Indicates Contamination Flag
    (\textit{koi\_fpflag\_ec})} indicante che il KOI condivide lo stesso periodo
    e l'epoca di un altro oggetto e che viene ritenuto essere il risultato di
    una "contaminazione" durante l'analisi.
\end{enumerate}

I dati presenti nelle successive colonne rappresentano invece i valori non 
booleani relativi alle effettive misurazioni (eventualmente arricchite con 
errori) di una certa osservazione. 

Tra questi dati si riconoscono varie categorie tra cui informazioni in merito al 
transito, ai dati della stella, etc$\ldots$

\section{Selezione iniziale dei dati}
Dopo una prima analisi del dataset si è scelto di effettuare alcune modifiche
con lo scopo di ottenere un formato più pulito e con le informazioni più utili 
ai fini dello studio.

\subsection{Esclusione delle osservazioni non ufficialmente confermate}
In merito ai record presenti, ci siamo concentrati unicamente sulle osservazioni per le quali 
è stata già assegnata una etichetta di tipo \textbf{CONFIRMED} o \textbf{FALSE POSITIVE}.
Tale scelta è stata presa poiché le restanti due classificazioni (\textit{CANDIDATE} o 
\textit{NOT DISPOSITIONED}) non permetterebbero una corretta analisi predittiva e la generazione
di modelli di machine learning adeguati, dato che non forniscono una relativa classificazione 
"sicura" della comunità scientifica in merito ai KOI. 
In seguito a tale considerazione, il problema si riduce ad una classificazione binaria.


\subsection{Esclusione delle caratteristiche correlate al target}
Si sono quindi rimosse le colonne aventi una relazione di dipendenza diretta
con il target. 
Per primi si sono rimosse le colonne relative agli identificativi assegnati 
unicamente alle osservazioni per cui il target è di tipo \textbf{CONFIRMED}, come ad
esempio l'ID ed il nome dell'esopianeta.
Si sono quindi esclusi i quattro attributi booleani precedentemente 
analizzati. Anch'essi sono fortemente correlati ad una specifica 
label, ad esempio il valore di \textit{koi\_fpflag\_ec} uguale a 1 indica con elevata 
probabilità che l'osservazione è frutto di una contaminazione proveniente da un altro oggetto,
il che indica quasi certamente che si tratta di un \textit{FALSE POSITIVE}.

Come è possibile facilmente intuire, non sono state prese in considerazione 
le colonne relative al \textit{koi\_pdisposition} e al relativo \textit{koi\_score} 
che nella pratica, ipotizzando la classificazione e il relativo punteggio di confidenza, 
svolgono le stesse stime da noi cercate.

Il dataset iniziale forniva inoltre l'attributo \textit{koi\_vet\_stat} in grado di 
certificare, nel caso di valore \textit{Done}, la conferma che la comunità scientifica è 
giunta ad una conclusione ufficiale su quel determinato KOI.
In seguito alla già citata esclusione delle osservazioni non etichettate come \textit{CONFIRMED} 
o \textit{FALSE POSITIVE}, i record rimasti sono quindi tutti rappresentati come ufficializzati.
Tale caratteristica, insieme alla data associata, è quindi stata rimossa.

Una successiva pulizia del dataset è avvenuta mediante l'eliminazione di tutte le colonne 
relative a caratteristiche per cui l'intero dataset presentava, per ogni record, 
un solo valore, compreso eventualmente il valore \textit{NULL}.\\

\subsection{Downsampling dei dati}
Anche in seguito all'esclusione delle caratteristiche sopracitate, il dataset mostrava un 
eccessivo sbilanciamento nei confronti dell'etichetta target. Più precisamente 
aveva una pendenza maggiore verso le etichette \textbf{FALSE POSITIVE}.

Si è quindi eseguita una operazione di \textit{downsample} in modo da ottenere una 
distribuzione equa tra le etichette target.

\begin{figure}[!htb]
    \begin{minipage}{0.48\textwidth}
      \centering
      \includegraphics[width=.7\linewidth]{../outputs/before_downsample_distr.pca.png}
      \caption{Distribuzione del target prima il downsampling}
      \label{Fig:Data1}
    \end{minipage}\hfill
    \begin{minipage}{0.48\textwidth}
      \centering
      \includegraphics[width=.7\linewidth]{../outputs/after_downsample_distr.pca.png}
      \caption{Distribuzione del target dopo il downsampling}
      \label{Fig:Data2}
    \end{minipage}
\end{figure}
% immagine matrice correlazione

\section{Principal Component Analysis}
In seguito alla rimozione delle caratteristiche dal dataset, si è quindi eseguita la PCA 
sullo stesso. Si è quindi scelto di tenere tutte le feature per le quali, con la tecnica della PCA, 
si ottiene un autovalore maggiore o uguale a 1. 
Queste scelta è dovuta al fatto che è una delle metriche maggiormente utilizzate.
\begin{figure}[H]
    \centering
    \includegraphics[width = \textwidth]{../outputs/hist_pca.png}
    \caption{Barplot dei contributi delle variabili rispetto alle dimensioni 
    della PCA. La linea rossa rappresenta il contributo medio atteso.}
\end{figure}
%\begin{figure}
%    \centering
%    \includegraphics[width = \textwidth]{../outputs/distr.pca.png}
%    \caption{Screeplot con gli autovalori ordinati dal maggiore al minore}
%\end{figure}
\begin{figure}[H]
    \centering
    \includegraphics[width = .7\textwidth]{../outputs/cumulative_variance.pca.png}
    \caption{Plot della varianza cumulata delle componenti della PCA}
    \label{fig:var}
\end{figure}
Alla fine della PCA sono state quindi selezionate 26 feature più significative,
a partire dalle iniziali 91, che verranno utilizzate per i modelli di machine 
learning. Queste numero di variabili comportano una varianza cumulata di 
$\sim 85\%$, come segnalato dalla linea verticale rossa in figura \ref{fig:var}
(mentre la linea verde segnala la varianza cumulata del $\sim 70\%$, un'altra
metrica spesso usata al posto dell'autovalore maggiore di uno).