\chapter{Risultati e considerazioni finali}
Possiamo quindi discutere nel dettaglio il confronto tra i vari modelli.\\
Innanzitutto è interessante confrontare in un unico grafico le curve ROC dei 
vari modelli, ricordando che si tratta delle curve relative alla classe 
"FALSE POSITIVE":
\begin{figure}[H]
    \centering
    \includegraphics[width = .5\textwidth]{../outputs/roc_full.comparison_FALSEPOSITIVE.png}
    \caption{Grafico con il confronto tra le ROC per classe "FALSE POSITIVE"}
\end{figure}
Da un primo sguardo a questo grafico si rileva una performance superiore, almeno
per questa metrica, per quanto riguarda i due modelli \textit{SVM}.\\
Proseguendo possiamo effettuare anche qualche indagine statistica.\\
Partento con un dotplot di confronto possiamo visualizzare anche qui come i due 
modelli \textit{SVM} presentino performance, dal punto di vista delle 
\textit{ROC}, superiori agli altri due modelli, soprattutto alla rete, che 
presenta un intervallo molto esteso di punteggio.
\begin{figure}[H]
    \centering
    \includegraphics[width = .5\textwidth]{../outputs/dotplot.comparison.png}
    \caption{Dotplot di confronto tra i modelli}
\end{figure}
Aggiungiamo quindi la visualizzazione, oltre che del punteggio relativo alle 
\textit{ROC}, anche della \textit{sensitività} e della \textit{specificità}:
\begin{figure}[H]
    \centering
    \includegraphics[width = .5\textwidth]{../outputs/bwplot.comparison.png}
    \caption{Bwplot di confronto tra i modelli}
\end{figure}
Dal punto di vista della \textit{sensitività} notiamo come i due modelli 
\textit{SVM} e il modello \textit{Naive Bayes} siano comparabili mentre possiamo 
anche qui notare il calo di performance in merito alla rete neurale. In merito 
alla \textit{specificità} notiamo un calo di performance anche in merito a bayes 
e la riconferma dell'inadeguatezza dalla rete neurale.\\
Visualizziamo quindi un confronto dei dati delle performance:
\begin{table}[H]
    \centering
    \begin{tabular}{c||c|c|c}
        & accuracy  & AUC FALSE POSITIVE \\
        \hline
        \hline
        SVM radiale & 0.9603  & 0.991\\
        SVM polinomiale & 0.9556 & 0.988\\
        Naive Bayes & 0.9245 & 0.982\\
        Rete neurale & 0.9151 & 0.972  
    \end{tabular}
\end{table}
Un ultimo confronto viene effettuato sui \textit{timing}. Visualizziamo la 
tabella relativa:
\begin{table}[H]
    \centering
    \begin{tabular}{c||c|c}
        & train completo & modello finale\\
        \hline
        \hline
        SVM radiale & 96.297 & 1.113 \\
        SVM polinomiale & 1129.200 & 0.952 \\
        Naive Bayes & 8.775 & 0.086 \\
        Rete neurale & 71.591 & 2.333 
    \end{tabular}
\end{table}
Notiamo come i tempi di train completo di \textit{SVM polinomiale} siano i più 
estesi. Questo è dovuto per lo più al tuning automatico fornito da 
\textit{caret} che lavora su tre parametri tutti e tre con tre valori 
possibili.\\
Interessante è notare come il modello di \textit{Naive Bayes} sia il più 
prestante, dal punto di vista del timing, in termini sia di train completo che 
del modello finale, fattore che,
in base allo studio prestazionale precedentemente effettuato, porta a ritenerlo 
comunque comparabile, in ipotetica fase di "produzione", ai due modelli 
\textit{SVM}, grazie a questo \textit{tradeoff} favorevole dei tempi di train.\\
Nel complesso quindi lo studio ha portato a concludere come i due modelli 
\textit{SVM} e il modello \textit{Naive Bayes} siano tutti e tre interessanti
a seconda dei diversi aspetti: performance delle predizioni e timing. D'altro
canto la rete neurale si è dimostrata non adatta per essere usata con questo
preciso dataset.