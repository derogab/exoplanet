\chapter{Risultati e considerazioni finali}
Abbiamo posto l'attenzione sull'etichetta \textit{FALSE POSITIVE}
per essere in grado di riconoscere con maggiore confidenza i KOI che 
non sono sicuramente esopieneti a discapito della confidenza nel 
riconoscimento dei KOI che sono sicuramente esopianeti. 

Tale scelta è puramente personale e dettata dal fatto che si 
vuole dare maggior importanza al riconoscimento dei KOI da scartare.
Così facendo si contempla concettualmente la possibilità per cui alcuni 
KOI classificati come \textit{CONFIRMED} possano essere riclassificati come 
\textit{FALSE POSITIVE} dopo un ulteriore studio scientifico.

Questa decisione diminuisce inoltre la possibilità che un esopianeta 
venga erroneamente classificato come \textit{FALSE POSITIVE} e scartato.

Ricordando che si tratta delle curve relative alla classe 
\textit{FALSE POSITIVE}, è interessante confrontare in un unico grafico le curve ROC dei 
vari modelli:
\begin{figure}[H]
    \centering
    \includegraphics[width = .5\textwidth]{../outputs/roc_full.comparison_FALSEPOSITIVE.png}
    \caption{Grafico con il confronto tra le ROC per classe "FALSE POSITIVE"}
\end{figure}

Da un primo sguardo al grafico si rileva che la miglior performance, almeno 
considerando questa metrica, risulta essere data dai modelli \textit{SVM} con
i due diversi kernel.

È stato inoltre eseguito un confronto dei dati delle performance:
\begin{table}[H]
    \centering
    \begin{tabular}{|c|c|c|c|c|c|}
        \hline
         & Accuracy  & Precision & Recall & F-Measure & AUC \\
        \hline
        \hline
        SVM radiale & 0.9603 & 0.9711 & 0.9522 & 0.9616 & 0.991 \\
        SVM polinomiale & 0.9556 & 0.9752 & 0.9388 & 0.9576 & 0.988 \\
        Naive Bayes & 0.9245 & 0.9614 & 0.8910 & 0.9249 & 0.980 \\
        Rete neurale & 0.8777 & 0.9325 & 0.8254 & 0.8757 & 0.938 \\
        \hline
    \end{tabular}
\end{table}

Dati i nostri scopi, cioè predirre il maggior numero di FALSE POSITIVE tra 
i KOI effettivamente con etichetta FALSE POSITIVE, ci possiamo concentrare sulla
misura di Recall, ovvero il rapporto tra numero di FALSE POSITIVE correttamente 
predetti e il numero totale di istanze con etichetta FALSE POSITIVE. 
Da questa misura possiamo notare che i risultati migliori si ottengono con i 
modelli SVM.

Proseguendo possiamo effettuare anche qualche indagine statistica.

Si analizza innanzitutto un \textbf{dotplot} che conferma, in 
seguito ad un semplice confronto, la superiorità delle performance 
di entrambi i modelli \textit{SVM} rispetto agli altri dal punto 
di vista delle \textit{ROC}.

Analizzando lo stesso grafico comprendiamo come il modello 
peggiore risulta essere quello relativo alla rete neurale, la 
quale presenta un intervallo di confidenza molto esteso.

\begin{figure}[H]
    \centering
    \includegraphics[width = .5\textwidth]{../outputs/dotplot.comparison.png}
    \caption{Dotplot di confronto tra i modelli}
\end{figure}

Aggiungiamo quindi la visualizzazione, oltre che del punteggio relativo alle 
\textit{ROC}, anche della \textit{sensitività} e della \textit{specificità}:
\begin{figure}[H]
    \centering
    \includegraphics[width = .5\textwidth]{../outputs/bwplot.comparison.png}
    \caption{Bwplot di confronto tra i modelli}
\end{figure}
Dal punto di vista della \textit{specificità} notiamo come i due modelli 
\textit{SVM} e il modello \textit{Naive Bayes} siano comparabili mentre possiamo 
anche qui notare il calo di performance in merito alla rete neurale. In merito 
alla \textit{sensitività} notiamo un calo di performance anche in merito a 
\textit{Naive Bayes} e la riconferma dell'inadeguatezza dalla rete neurale,
soprattutto considerando che la \textit{sensitività}, come precedentemente 
anticipato, è al centro di questo studio.\\

Un ultimo confronto viene effettuato sui \textit{timing}. Visualizziamo la 
tabella relativa:
\begin{table}[H]
    \centering
    \begin{tabular}{c||c|c}
        & train completo & modello finale\\
        \hline
        \hline
        SVM radiale & 89.118 & 0.936 \\
        SVM polinomiale & 1215.629 & 0.838 \\
        Naive Bayes & 7.622 & 0.102\\
        Rete neurale & 65.747 & 1.524
    \end{tabular}
\end{table}
Notiamo come i tempi di train completo di \textit{SVM polinomiale} siano i più 
estesi. Questo è dovuto per lo più al tuning automatico fornito da 
\textit{caret} che lavora su tre parametri tutti e tre con tre valori 
possibili.\\
Interessante è notare come il modello di \textit{Naive Bayes} sia il più 
prestante, dal punto di vista del timing, in termini sia di train completo che 
del modello finale.\\
Nel complesso quindi lo studio ha portato a concludere come i due modelli 
\textit{SVM} e il modello \textit{Naive Bayes} siano tutti e tre interessanti
a seconda dei diversi aspetti: performance delle predizioni e timing. D'altro
canto la rete neurale si è dimostrata non adatta per essere usata con questo
preciso dataset.\\
In definitiva, ad una prima analisi, valutando il tradeoff tra costo di training
e performance di predizione, il modello \textit{SVM} con kernel radiale 
sembrerebbe essere la soluzione più interessante da usare in un ipotetico
contesto di "produzione" anche se si segnala che non si hanno veri e propri
discriminanti per preferire un modello rispetto ad un altro, ad esclusione delle
reti neurali che non possono essere preferite in alcun caso.